%!TEX root = proposal.tex

\section{Related Work}

\chenhao{cite more CHI work here}

\citep{birnholtz2013write}

\para{Prior work.} We conducted a pilot study with 18 participants in \citet{clark+etal18} on slogan writing.
\chenhao{feel free to shorten this part.}
% Slogan writing presents a challenge to writers that is similar to humor generation: to generate a concise, memorable, and powerful sentence that is representative of the object it promotes and matches the intention of the authors. 
% Slogans are used for promotion in a variety of settings, ranging from organizing a social movement to promoting a product. 
% Success in machine-in-the-loop systems for slogan writing and humor generation can lead to potential paradigm changes in many domains of creative writing.
% a system that supports slogan writing could be extended to several related tasks that prioritize catchy and succinct language.
We asked all participants to write slogans for three scenarios.
Half of them wrote with a machine in the loop ({\em treatment}), while the other half wrote on their own ({\em control}).
We asked participants to assess the writing process and their final slogans, and answer open-ended questions in an interview.
We also collected third-party evaluations of each slogan from Amazon Mechanical Turk.
% The machine learning system adopts an iterative design that is not intrusive and uses pull-based suggestions.
One key difference from this proposal is that the machine learning system generates {\bf full-sentence suggestions}.
Here we highlight the most relevant findings.
% Our results show that (for space reasons, including only three findings):


\begin{enumerate}[leftmargin=*,noitemsep,topsep=0pt,parsep=0pt,partopsep=0pt]
    \item Users generally found the writing process more engaging and more fun with a machine in the loop.
    \item Users with a machine in the loop generally rated their own slogans better than the control group, but there is no difference in third-party evaluation.
    \item Users gave low ratings to the suggestions in every aspect, especially regarding relevance (see \figref{fig:user}).
\end{enumerate}

% We draw insights from quantitative ratings from participants and qualitative interviews with users to improve our machine learning systems that generate suggestions.
% To motivate our proposed ideas in this proposal, we give a brief overview of the computational model in the slogan writing machine learning system.
Overall, our preliminary findings suggest that machine-in-the-loop systems for creative writing are promising, however, the computational model requires improvement. 
In fact, many participants point out in the interviews that keyword suggestions may be much more useful.