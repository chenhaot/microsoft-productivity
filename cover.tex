%!TEX root = proposal.tex





% His work has been covered by many news media outlets, such as the New York Times and the Washington Post. He also won an NSF CRII award, a Facebook fellowship, and a Yahoo! Key Scientific Challenges award. \chenhao{probably should shorten further.}



\para{Abstract.} 
% Collaborative writing plays a central role in our professional life.
This project investigates how intelligent agents can support collorative writing, a ubiquitous activity in human productivity.
To do that, we develop computational algorithms to provide two types of suggestions: collaboration suggestions and content suggestions.
We will further explore the design space of how these suggestions will be provided and integrated in the collaborative writing process.
We consider diverse writing tasks, including potentially working with real users to create content on Wikipedia.
In addition to two-person synchronous collaborations, we will also examine five-person teams to pave the road for future work on collaborative writing.
Success in this project will promote the vision of human-centered AI and enhance the productivity in a wide range of collaborative settings.

\para{Biographical information and contact information.}
Chenhao Tan is an assistant professor of computer science at University of Colorado Boulder. He obtained his PhD degree in the Department of Computer Science at Cornell University and bachelor's degrees in computer science and in economics from Tsinghua University. Prior to joining CU Boulder, he spent a year at the University of Washington as a postdoc. His research interests include human-centered AI, natural language processing, and computational social science. 
% He has published papers primarily at ACL and WWW, and also at KDD, WSDM, ICWSM, etc.
He has served as area chairs for NAACL, EMNLP, and AAAI.
He will lead the design of computational algorithms and models required to offer suggestions in the collaborative writing process, building on prior work on creative writing with individuals \citep{clark+etal18}.

Ming Yin is an assistant professor of computer science at Purdue University. She obtained his PhD degree in Computer Science at Harvard University, and bachelor's degree from Tsinghua University. Before joining Purdue, she spent a year at Microsoft Research New York City as a Postdoctoral Researcher. Her research interests include social computing and crowdsourcing, human-AI interaction, and computational social science. She has served as associate chair or program committee members for conferences like CHI, WWW, and AAAI. She will lead the design and implementation of behavioral experiments in this proposal.