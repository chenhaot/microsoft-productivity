%!TEX root = proposal.tex

\section{Experimental Design}

In this section, we outline the experiments we plan to conduct in this project. Specifically, we envision a set of 4 behavioral experiments to advance the understandings on how to better design machine-generated suggestions to enhance the productivity in collaborative writing. We first design three experiments (\secref{sec:exp1}---\ref{sec:exp3}) to examine the effectiveness of various collaboration suggestions and content suggestions in the context of collaborative writing within a team of two people. We then discuss in \secref{sec:exp4} the additional challenges and opportunities for experimental studies on machine-in-the-loop collaborative writing when a significantly larger number of individuals participate in the writing process.

\subsection{Experiment 1: Examining the Effectiveness of Collaboration Suggestions}
\label{sec:exp1}

As discussed in \secref{sec:workflow}, multiple types of suggestions can be provided to a team of writers to encourage more effective collaborations between them. The goal of the first experiment is thus to understand, for a team of two writers, what the most effective way to provide collaborative suggestions is. We start our experiments with the smallest possible ``team'' (i.e., a team of two people) as it is logistically most feasible to conduct experiments with two-person teams. We construct a design space of collaboration suggestions based on different dimensions on which these suggestions can focus:
\begin{itemize}
\item {\bf Flow of collaboration}: This factor concerns whether and how suggestions are provided on the decomposition of the grand writing task into subtasks. Possible different levels of suggestions for this factor include: (1) {\em none}, that is, the team will not be provided with any suggestion on the need of dividing the task into subsequent steps; (2) {\em encourage workflow-related discussions}, that is, suggestions will be provided at the beginning of the writing process which recommend team members to discuss how to divide the task into a sequence of subtasks; (3) {\em define workflow}, that is, the team will receive suggestions asking them to follow a particular sequence of steps to complete the writing task (e.g., brainstorm and rank ideas--expand top ideas into candidate slogans--refine wordings). This workflow can be obtained by conducting data mining on existing logs of collaborative writing processes to identify common patterns.  
\item {\bf Division of labor}: This factor concerns whether and how suggestions are provided on the division of labor among the two writers. Possible different levels of suggestions for this factor include: (1) {\em none}, the team does not receive any suggestion on the need of splitting the workload among the team members; (2) {\em encourage work-division-related discussions}, in which suggestions will be provided to the team (potentially within each subtask) to encourage the workload division discussion within the team; (3) {\em divide the workload}, which means that machines will directly provide suggestions on how to split the work among team members, which by itself can take multiple formats. The most simple version of such suggestions can be an equal split between the team members (e.g., each person generates 10 ideas / writes 5 slogans). More sophisticated suggestions can be designed after each team member's skills and strengths are learned from historic data, and machines can even monitor the progress of each team member during the writing process and dynamically update the division.
\item {\bf Pace of work}:
\end{itemize}

\subsection{Experiment 2: Examining the Effectiveness of Content Suggestions among Two People}
\label{sec:exp2}

\subsection{Experiment 3: Combining Collaboration Suggestions and Content Suggestions}
\label{sec:exp3}


\subsection{From Dyad to Team}
\label{sec:exp4}