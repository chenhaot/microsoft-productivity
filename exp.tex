%!TEX root = proposal.tex

\section{Experimental Design}

In this section, we outline the experiments we plan to conduct in this project to advance the understandings on how to better incorporate machines into the loop to enhance the productivity in collaborative writing. We first design three experiments (\secref{sec:exp1}---\ref{sec:exp3}) to examine the effectiveness of various collaboration suggestions and content suggestions in the context of collaborative writing within a team of two people. We then discuss in \secref{sec:exp4} the additional challenges and opportunities for experimental studies on machine-in-the-loop collaborative writing when a significantly larger number of individuals participate in the writing process. We'll recruit both college students and crowd participants from online crowdsourcing platforms like Amazon's Mechanical Turk (MTurk) to participate in our experiment. We also plan to explore the possibility of field studies with Wikipedia contributors to understand the real-world impact of machine-generated suggestions in collaborative writing.

\subsection{Experiment 1: Examining the Effectiveness of Collaboration Suggestions}
\label{sec:exp1}

%As discussed in \secref{sec:workflow}, multiple types of suggestions can be provided to a team of writers to encourage more effective collaborations between them. 
The goal of the first experiment is to understand, within a team of two writers, what the most effective way to provide collaborative suggestions is. We start our experiments with the smallest possible ``team'' (i.e., a team of two people) as it is logistically most feasible to conduct experiments with two-person teams. 

As discussed in \ref{sec:workflow}, the design space of collaboration suggestions includes factors like the type of suggestions, the content of suggestions, and the timing of suggestions. To simplify the experiment, we first fix the timing of all types of suggestions based on some pre-defined heuristics and explore what the most effective combinations of collaboration suggestion type and content are. Specifically, we consider a factorial design that experimental treatments differ along three independent variables---the provision of collaboration flow suggestions, the provision of division of labor/work sharing suggestions, and the provision of work pace suggestions. Each independent variable can take multiple levels to reflect different content of the suggestion, and we currently consider three levels for each independent variable:
\begin{itemize}[leftmargin=*]
\item {\em None}: No suggestion is provided to the team on the given aspect of collaboration (i.e., flow of collaboration, division of labor/work sharing, pace of work).
\item {\em Discussion encouraging suggestion}: The machine simply provides a suggestion that nudge the team into discussing the given aspect of collaboration. %(e.g., how to divide the task into subtasks, how to divide the workload, and whether adjustment of working pace is needed). 
Such suggestion can draw the team's attention to specific aspects in collaboration while leaving the freedom to the team on making specific decisions on those aspects.
\item {\em Actionable suggestion}: The machine provides actionable steps to the team on the given aspect of collaboration, such as asking the team to follow a particular workflow, directly proposing a split of work between team members, and actively suggesting the team to speed up, slow down, or take a break. While such actionable suggestions may provide concrete guidance on how the team should proceed to achieve their shared goal, it is unclear how the team will interpret these suggestions (e.g., are they decreasing the team's control to their work?) and whether they will follow them. 
\end{itemize}

\ignore{
\begin{itemize}[leftmargin=*]
\item {\bf Suggestion type}: The type of suggestions provided to the team. Possible different levels of suggestions 
\end{itemize}

\begin{itemize}
\item {\bf Flow of collaboration}: This factor concerns whether and how suggestions are provided on the decomposition of the grand writing task into subtasks. Possible different levels of suggestions for this factor include: (1) {\em none}, that is, the team will not be provided with any suggestion on the need of dividing the task into subsequent steps; (2) {\em encourage workflow-related discussions}, that is, suggestions will be provided at the beginning of the writing process which recommend team members to discuss how to divide the task into a sequence of subtasks; (3) {\em define workflow}, that is, the team will receive suggestions asking them to follow a particular sequence of steps to complete the writing task (e.g., brainstorm and rank ideas--expand top ideas into candidate slogans--refine wordings). This workflow can be obtained by conducting data mining on existing logs of collaborative writing processes to identify common patterns.  
\item {\bf Division of labor}: This factor concerns whether and how suggestions are provided on the division of labor among the two writers. Possible different levels of suggestions for this factor include: (1) {\em none}, the team does not receive any suggestion on the need of splitting the workload among the team members; (2) {\em encourage work-division-related discussions}, in which suggestions will be provided to the team (potentially within each subtask) to encourage the workload division discussion within the team; (3) {\em divide the workload}, which means that machines will directly provide suggestions on how to split the work among team members, which by itself can take multiple formats. The most simple version of such suggestions can be an equal split between the team members (e.g., each person generates 10 ideas/writes 5 slogans). More sophisticated suggestions can be designed after each team member's skills and strengths are learned from historic data, and machines can even monitor the progress of each team member during the writing process and dynamically update the division.
\item {\bf Pace of work}: This factor concerns whether and how suggestions are provided on the pace of the collaborative work. Possible different levels of suggestions for this factor include: (1) {\em none}, no suggestions are provided to the team on whether and how to control the pace of the work; (2) {\em encourage reflections on work pace}, where at multiple time points during the writing process the team will be suggested to reflect on what they have accomplished and what need to be accomplished, and adjust their working speed accordingly; (3) {\em control pace}, where the machines will provide direct suggestions asking the team to speed up, slow down or take a break. These suggestions can be designed through both fixed rules (e.g., take a break after a fixed amount of time) and actively reacting to current work status of the team (e.g., suggest team to speed up if the observed activity level is very low for a long time). 
\end{itemize}
}

Both the quality of collaborative writing product (e.g., the third-party evaluation on the quality of a slogan) and the amount of time it takes to produce it will be used as our dependent variables to quantify productivity. %As a result, Experiment 1 enables us to examine the effect of each type of collaboration suggestions on influencing productivity. Specifically, 
We make the following hypotheses:
\begin{itemize}
\item{\bf H1}: Providing some suggestions on the flow of collaboration between team members improves the productivity in collaborative writing.
\item{\bf H2}: Providing some suggestions on the division of labor between team members improves the productivity in collaborative writing.
\item{\bf H3}: Providing some suggestions on the pace of work for a team improves the productivity in collaborative writing.
\end{itemize}
In addition, as open questions, we ask: (1) is it more effective to provide discussion encouraging suggestions or actionable suggestions? and (2) what is the most effective combination of collaboration strategies, and does it depend on the property of the writing task?

Finally, with the best combinations of suggestion type and content identified, we are interested in further comparing the effectiveness of these suggestions when they are provided at different timing (e.g., the timing can be decided based on different heuristics or computational methods that predict how interruptive/noticeable a suggestion is when provided at different times) and thus understand best timing of collaboration suggestions.

\subsection{Experiment 2: Examining the Effectiveness of Content Suggestions among Two People}
\label{sec:exp2}
In addition to providing suggestions on how a team of writers should collaborate with each other, Section~\ref{sec:content} highlights the opportunity of utilizing machine intelligence to help the team enhance the quality of the content they write. Thus, the goal of the second experiment is thus to conduct user studies to examine the effectiveness of these content suggestions in real-world two-person writing scenarios. In particular, we hypothesize that:
\begin{itemize}
\item{\bf H4}: Providing brainstorm suggestions improves the productivity in collaborative writing.
\item{\bf H5}: Providing lexical suggestions improves the productivity in collaborative writing.
\end{itemize}

To test these hypotheses, we can adopt a $2\times 2$ design where treatments differ along two factors: (1) the provision of brainstorm suggestions (with two levels: yes or no), and (2) the provision of lexical suggestions (with two levels: yes or no). Similar as before, we will use the quality of writing product and the amount of time used to produce the writing as our dependent variables to characterize productivity. Intuitively, the higher the writing quality and/or the faster the team can finish the writing, the more productive the team is. Moreover, if multiple kinds of brainstorm suggestions and/or lexical suggestions can be designed (e.g., lexical suggestions that suggest synonyms vs. lexical suggestions that allow users to indicate the specific dimensions of word properties that they are looking for), our experiment can be easily extended to include more levels for each independent variable to explore which types of brainstorm/lexical suggestions are more effective in improving productivity.

\subsection{Experiment 3: Combining Collaboration Suggestions and Content Suggestions}
\label{sec:exp3}

In the third experiment, we plan to study how collaboration suggestions and content suggestions interplay with each other to influence productivity. \my{Do not know what to write yet...} 


\subsection{From Dyad to Team}
\label{sec:exp4}
Finally, we are also interested in extending the experiments into a more realistic setting where a significantly larger number of members (e.g., 5 or more people) are working together in the collaborative writing process. Compared to the previous experiments that focus on two-person teams, larger team introduces a number of factors that can potentially moderate the effects of collaboration suggestions and content suggestions on influencing productivity, such as:
\begin{itemize}
\item{\bf Authority structure}: The team can have a flat structure where every member has equal right to speak or a clear hierarchical structure in which some members are subordinates of other members.
\item{\bf Communication structure}: The team can vary in who can/are willing to talk to whom, such that the communication network within the team can take different structure (e.g., closely-knit group, star structure, two subnetworks with only a few connections between subnetworks, etc.) 
\item{\bf Diversity}: The team can be composed of members who are similar with each or different from one another. The similarity of team members can be defined along many dimensions, such as demographics, cognitive styles, skills, etc.
\end{itemize}

Thus, the goal of our final experiment is to explore whether and how the effects of collaboration and content suggestions on productivity vary for teams with different properties, such as different authority structure, communication structure, and different kinds and levels of diversity. To this end, we can conduct the experiment in a way that teams are constructed with designed properties, such as assigning roles for each team members to mimic authority structure, limiting the set of team members that one can communicate with to reflect different communication structure, and composing teams by mixing up similar or different members together to create different levels of diversity. Then, we can apply similar factorial design to different teams and measure the productivity of team in each treatment.