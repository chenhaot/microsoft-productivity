%!TEX root = proposal.tex

\section{Experimental Design}

In this section, we outline the experiments we plan to conduct in this project. Specifically, we envision a set of 4 behavioral experiments to advance the understandings on how to better design machine-generated suggestions to enhance the productivity in collaborative writing. We first design three experiments (\secref{sec:exp1}---\ref{sec:exp3}) to examine the effectiveness of various collaboration suggestions and content suggestions in the context of collaborative writing within a team of two people. We then discuss in \secref{sec:exp4} the additional challenges and opportunities for experimental studies on machine-in-the-loop collaborative writing when a significantly larger number of individuals participate in the writing process.

\subsection{Experiment 1: Examining the Effectiveness of Collaboration Suggestions}
\label{sec:exp1}

As discussed in \secref{sec:workflow}, multiple types of suggestions can be provided to a team of writers to encourage more effective collaborations between them. The goal of the first experiment is thus to understand, for a team of two writers, what the most effective way to provide collaborative suggestions is. We start our experiments with the smallest possible ``team'' (i.e., a team of two people) as it is logistically most feasible to conduct experiments with two-person teams. We construct a design space of collaboration suggestions based on different dimensions on which these suggestions can focus:
\begin{itemize}
\item {\bf Flow of collaboration}: This factor concerns whether and how suggestions are provided on the decomposition of the grand writing task into subtasks. Possible different levels of suggestions for this factor include: (1) {\em none}, that is, the team will not be provided with any suggestion on the need of dividing the task into subsequent steps; (2) {\em encourage workflow-related discussions}, that is, suggestions will be provided at the beginning of the writing process which recommend team members to discuss how to divide the task into a sequence of subtasks; (3) {\em define workflow}, that is, the team will receive suggestions asking them to follow a particular sequence of steps to complete the writing task (e.g., brainstorm and rank ideas--expand top ideas into candidate slogans--refine wordings). This workflow can be obtained by conducting data mining on existing logs of collaborative writing processes to identify common patterns.  
\item {\bf Division of labor}: This factor concerns whether and how suggestions are provided on the division of labor among the two writers. Possible different levels of suggestions for this factor include: (1) {\em none}, the team does not receive any suggestion on the need of splitting the workload among the team members; (2) {\em encourage work-division-related discussions}, in which suggestions will be provided to the team (potentially within each subtask) to encourage the workload division discussion within the team; (3) {\em divide the workload}, which means that machines will directly provide suggestions on how to split the work among team members, which by itself can take multiple formats. The most simple version of such suggestions can be an equal split between the team members (e.g., each person generates 10 ideas/writes 5 slogans). More sophisticated suggestions can be designed after each team member's skills and strengths are learned from historic data, and machines can even monitor the progress of each team member during the writing process and dynamically update the division.
\item {\bf Pace of work}: This factor concerns whether and how suggestions are provided on the pace of the collaborative work. Possible different levels of suggestions for this factor include: (1) {\em none}, no suggestions are provided to the team on whether and how to control the pace of the work; (2) {\em encourage reflections on work pace}, where at multiple time points during the writing process the team will be suggested to reflect on what they have accomplished and what need to be accomplished, and adjust their working speed accordingly; (3) {\em control pace}, where the machines will provide direct suggestions asking the team to speed up, slow down or take a break. These suggestions can be designed through both fixed rules (e.g., take a break after a fixed amount of time) and actively reacting to current work status of the team (e.g., suggest team to speed up if the observed activity level is very low for a long time). 
\end{itemize}

In Experiment 1, we plan to adopt a factorial design where each of the above factors in the design space of collaboration suggestions will serve as the independent variables of our experiment, while we will measure both the quality of collaborative writing product (e.g., \my{How do we typically measure the quality of slogan?}) and the amount of time it takes to produce it as our dependent variables to quantify productivity. As a result, Experiment 1 enables us to examine the effect of each type of collaboration suggestions on influencing productivity. Specifically, we make the following hypotheses:
\begin{itemize}
\item{\bf H1}: Providing suggestions on the flow of collaboration between team members improves the productivity in collaborative writing.
\item{\bf H2}: Providing suggestions on the division of labor between team members improves the productivity in collaborative writing.
\item{\bf H3}: Providing suggestions on the pace of work for a team improves the productivity in collaborative writing.
\end{itemize}
Finally, as open questions, we ask: (1) is it more effective to provide suggestions to encourage discussions about flow of collaboration/division of labor/pace of work by team members themselves, or provide direct, actionable suggestions on these aspects to team members? and (2) what is the most effective combination of collaboration strategies, and does it depend on the property of the writing task?

\subsection{Experiment 2: Examining the Effectiveness of Content Suggestions among Two People}
\label{sec:exp2}
In addition to providing suggestions on how a team of writers should collaborate with each other, Section~\ref{sec:content} highlights the opportunity of utilizing machine intelligence to help the team enhance the quality of the content they write. Thus, the goal of the second experiment is thus to conduct user studies to examine the effectiveness of these content suggestions in real-world scenarios. In particular, we hypothesize that:
\begin{itemize}
\item{\bf H4}: Providing brainstorm suggestions improves the productivity in collaborative writing.
\item{\bf H5}: Providing lexical suggestions improves the productivity in collaborative writing.
\end{itemize}

To test these hypotheses, we can adopt a $2\times 2$ design where treatments differ along two factors: (1) the provision of brainstorm suggestions (with two levels: yes or no), and (2) the provision of lexical suggestions (with two levels: yes or no). Similar as before, we will use the quality of writing product and the amount of time used to produce the writing as our dependent variables to characterize productivity. Intuitively, the higher the writing quality and/or the faster the team can finish the writing, the more productive the team is. Moreover, if multiple kind of brainstorm suggestions and/or lexical suggestions can be designed (e.g., lexical suggestion that suggest synonyms vs. lexical suggestions that allow users to indicate the specific dimensions of word properties that they are looking for), our experiment can be easily extended to include more levels for each independent variable to explore which types of brainstorm/lexical suggestions are more effective in improving productivity.

\subsection{Experiment 3: Combining Collaboration Suggestions and Content Suggestions}
\label{sec:exp3}


\subsection{From Dyad to Team}
\label{sec:exp4}