%!TEX root = proposal.tex

\section{Experimental Design}

In this section, we outline the experiments we plan to conduct in this project. Specifically, we envision a set of 4 behavioral experiments to advance the understandings on how to better design machine-generated suggestions to enhance the productivity in collaborative writing. We first design three experiments (\secref{sec:exp1}---\ref{sec:exp3}) to examine the effectiveness of various collaboration suggestions and content suggestions in the context of collaborative writing within a team of two people. We then discuss in \secref{sec:exp4} the additional challenges and opportunities for experimental studies on machine-in-the-loop collaborative writing when a significantly larger number of individuals participate in the writing process.

\subsection{Experiment 1: Examining the Effectiveness of Collaboration Suggestions among Two People}
\label{sec:exp1}

As discussed in \secref{sec:workflow}, multiple types of suggestions can be provided to a team of writers to encourage more effective collaboration between them. The goal of the first experiment is thus to understand, for a team of two writers, what the most effective way to provide collaborative suggestions is. We construct a design space of collaboration suggestions based on different dimensions on which collaboration suggestions can be provided:
\begin{itemize}
\item {\bf Collaboration structure}: This factor concerns whether and how suggestions are provided on the decomposition of the grand writing task into subtasks. For example, in the baseline condition, no suggestion will be provided on how to divide the task into subsequent steps. In other conditions, however, we can guide the team to follow different workflows for completing the writing task (e.g., brainstorm and rank ideas--expand top ideas into candidate slogans--refine wordings). To obtain different workflows of the writing task that we can experiment with, we can either conduct data mining on existing logs of collaborative writing processes to identify common patterns, or outsource the task to the crowd to solicit diverse ways for dividing the task.  
\item {\bf Division of labor}:
\item {\bf Pace of work}:
\end{itemize}

\subsection{Experiment 2: Examining the Effectiveness of Content Suggestions among Two People}
\label{sec:exp2}

\subsection{Experiment 3: Combining Collaboration Suggestions and Content Suggestions}
\label{sec:exp3}


\subsection{From Dyad to Team}
\label{sec:exp4}